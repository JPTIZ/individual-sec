%-------------------------------------------------------------------------------
\documentclass{article}

%-------------------------------------------------------------------------------
% Packages

% Translation
\usepackage[brazilian]{babel}

% General
\usepackage{amsmath}
\usepackage{environ}
\usepackage[margin=1in]{geometry}
\usepackage{xcolor}

% Other
\usepackage{hyperref}
\usepackage[brazilian]{cleveref} % Must be after hyperref

%-------------------------------------------------------------------------------
% User-commands
\newcommand{\todo}[1]{{\color{red}{#1}}}

\NewEnviron{superframe}{%
    \begin{center}
        \fbox{\setlength{\fboxsep}{1em}\fbox{\parbox{5.5in}{%
            \BODY{}
        }}}
    \end{center}
}

\NewEnviron{answer}{%
    \begin{samepage}
        \begin{solution}
            \BODY{}
        \end{solution}
    \end{samepage}
}

%-------------------------------------------------------------------------------
% Project configs
\title{
    Segurança em Computação \\
    Trabalho Individual I
}
\author{João Paulo Taylor Ienczak Zanette}
\date{\today}

%-------------------------------------------------------------------------------
\begin{document}
    \maketitle{}

    \begin{superframe}
        O objetivo deste trabalho é que você exercite a forma de pensar de um
        profissional de segurança em computação. Por isso voce deve produzir um
        relatório analisando algum sistema que você usa corriqueiramente quanto
        a sua segurança, lembre de avaliar pelo menos:

        \begin{itemize}
            \item Ativos?
            \item Adversários?
            \item Gerenciamento de Risco?
            \item Contra medidas?
            \item Custo/Beneficio?
        \end{itemize}

        A avaliação deste trabalho vai exigir que você escolha um sistema com
        um grau de complexidade pelo menos médio. Sistemas de baixa
        complexidade implicam em nota máxima de 60\%. Cada item avaliado
        corresponde a 20\% da nota final do trabalho. A avaliação de cada item
        requer uma discussão embasada nos fundamentos discutidos em aula e um
        grau de profundidade médio (pelo menos 3 parágrafos por item avaliado),
        exigindo que o alunos discorra coerentemente sobre o tópico.

        A entrega deve ser feita obrigatoriamente em PDF (o sistema não vai
        deixar entregar de outra forma). Atrasos sofrerão punição de 50\% da
        nota independente da data de entrega.
    \end{superframe}

    \section{Sobre o Sistema Escolhido}

    O sistema escolhido para este trabalho foi o \textbf{Sistema de Controle
    Acadêmico da Graduação} (CAGR), visto que, enquanto sistema da graduação,
    há bastante uso especialmente em períodos de matrícula, já que é necessário
    que esta seja feita nele, e ele é quem apresenta quais matérias o aluno
    conseguiu ou não, a grade de horários com salas em cada dia, professores,
    turmas e suas vagas, dentre outras informações de uso comum dos
    universitários da UFSC\@. Esta análise será estendida também ao
    \textbf{Fórum da Graduação}, um subsistema do CAGR para discussão entre
    alunos, com tópicos separados em fóruns de disciplinas ou do curso.

    Toda a análise neste relatório será feita \textbf{do ponto de vista do
    aluno}.

    \section{Análise}

    \subsection{Ativos}

    \subsubsection{Geral}

    Por se tratar do sistema oficial da graduação, há informações bastante
    importantes (institucionalmente) de cada aluno. Nem todas elas, porém, tem
    um valor monetário direto. Para boa parte delas, o uso indevido acarreta em
    problemas administrativos e transtornos para os alunos. Sumarizando a parte
    principal do CAGR\@:

    \begin{itemize}
        \item Dados cadastrais (detalhado
            nas~\Cref{sub:data-inst,sub:data-pers});
        \item Pedidos de matrícula (inclusões, exclusões e alterações);
        \item E-mail e senha\footnote{Não são institucionais, mas são
            essenciais para ter acesso para ler e alterar todas as outras
            informações.}
    \end{itemize}

    Dos pedidos de matrícula, é importante que o aluno não tenha sua matricula
    fora da regularidade por conta de alterações indesejadas nos pedidos de
    suas disciplinas. Além ainda da regularidade, vale apontar o transtorno de
    ser redirecionado para turmas e disciplinas que não compactuam com os
    objetivos do aluno, incluindo casos como estar sobrecarregado de
    disciplinas contra a própria vontade. Desfazer/corrigir essas mudanças
    demandaria ao aluno tratar diretamente com autoridades como o Coordenador
    de Curso e principalmente da boa vontade de professores entenderem a
    situação e poderem tomar medidas como aumento do número de vagas em suas
    turmas para que o aluno pudesse cursar as disciplinas que havia
    voluntariamente pedido. Vale apontar que há casos em que o professor não
    teria como alimentar a demanda de vagas, uma vez que necessita de espaço e
    tempo em aula para coordenar a turma. Ou seja, o pedido de matrícula de um
    aluno deve ser protegido.

    Quanto ao e-mail e senha, não é necessário elaborar muito sobre sua
    necessidade de proteção: o não recebimento de e-mails (por conta de um
    invasor configurar um e-mail que não seja algum de acesso do aluno) pode
    levar o aluno a perder informações importantíssimas que prejudicariam sua
    vida acadêmica (perda de provas por alterações de data, ou entregas de
    trabalho, etc.). A senha seria ainda mais catastrófica, uma vez que um
    atacante tendo acesso a ela é possível fazer todas as alterações
    indesejadas citadas até o momento, e ainda configurar uma senha que não é a
    do aluno (fazendo com que este tenha que resolver o caso com a instituição,
    gerando situações inconvenientes). Há de lembrar ainda a possibilidade da
    senha configurada ser a mesma de outras contas do aluno, outras ainda mais
    pessoais (plataformas de \textit{chat} instantâneo, repositório de arquivos
    na nuvem --- e.g. Google Drive ---, etc.).

    Vale lembrar que, se noticiada publicamente, uma invasão dessas mancharia a
    visão do público sobre a instituição, ainda mais sendo uma instituição
    pública em um país em que é comum a descrença quanto a recursos
    relacionados ao governo.

    \subsubsection{Dados cadastrais institucionais~\label{sub:data-inst}}

    Dados institucionais incluem:

    \begin{itemize}
        \item E-mail institucional;
        \item Currículo do aluno (notas e frequência nas disciplinas em cada
            semestre).
    \end{itemize}

    \subsubsection{Dados cadastrais pessoais~\label{sub:data-pers}}

    Dados pessoais incluem:

    \begin{itemize}
        \item CPF e RG\@;
        \item Tipo sanguíneo;
        \item Cidade natural;
        \item Nome dos pais;
        \item Procedência escolar:
            \begin{itemize}
                \item Ano de conclusão;
                \item Escola;
                \item Curso (quando há técnico integrado, por exemplo);
                \item Cidade em que cursou.
            \end{itemize}
    \end{itemize}

    \subsection{Adversários}

    Um dos interessados nos dados guardados no CAGR pode ser um inimigo pessoal
    de um aluno. Tendo acesso aos dados, pode acontecer alguma forma de
    sabotagem ou humilhação pública. Por exemplo, no semestre 2015.2 do curso
    de Ciência da Computação, alguém obteve acesso ao CAGR de um aluno sem o
    consentimento dele (que possivelmente esqueceu algum computador da
    instituição logado) e postou no fórum da graduação (em que os professores
    possuem acesso e potencialmente recebem e-mail do que é postado nele)
    intitulado ``Trote Opressor e Ofensivo'', que fazia uma falsa denúncia de
    homofobia e bullying no nome do tal aluno. Uma postagem dessas pode por em
    jogo, além da imagem dos veteranos do curso, futuros eventos de recepção
    para calouros.

    Atualmente também são valiosas informações para diferentes empresas e
    instituições. Pode ser considerado um adversário alguém interessado em
    roubo de informações seja para vender ou para uso direto. São informações
    importantes para quem busca contratar recém-formados ou graduandos, por
    exemplo.

    Outros adversários podem envolver tanto quem queira sabotar resultados de
    processos seletivos (por exemplo, para pós-graduação, alterando o histórico
    escolar de algum aluno) quanto \todo{quem mesmo?}.

    \subsection{Gerenciamento de Risco}

    Considerando que o CAGR é um sistema web, o custo de se realizar um ataque
    é potencialmente baixo (desconsiderando possíveis consequências jurídicas).
    Por exemplo, um \textit{SQL Injection} teria apenas o custo de conhecer a
    estrutura interna dos bancos de dados do sistema e achar qual campo não é
    sanitizado. Quanto ataques via \textit{buffer-overflow}, como o CAGR é
    feito em Java, ele depende quase totalmente da implementação das
    bibliotecas utilizadas.

    Outro detalhe, o CAGR é um sistema direcionado à instituição UFSC, e
    portanto \emph{potencialmente} não tem \textit{mirrors} espalhados pelo
    país. Logo, o custo de se realizar um ataque é compensado pelo ganho de se
    ter acesso à única fonte daquelas informações (salvo \textit{backups}).
    Isso significa tanto no fato de que obstrução de informações tem maior
    impacto quanto no de que, tendo acesso a uma informação, potencialmente se
    tem acesso a todas as outras.

    Além disso, centralização pode facilitar ataques como \textit{Denial of
    Service}. Esse ataque poderia não trazer estragos muito grandes, já que o
    servidor poderia ser religado pouco depois. Porém, isso poderia gerar
    transtornos aos alunos (tempo de matrícula, emissão de atestado de
    matrícula, etc.) e deve-se protegido contra, ainda que não seja um impacto
    ireto na instituição mas sim em seus clientes.

    \subsection{Contra medidas}

    De contra-medidas relacionadas a \textit{software}, a primeira a se pensar,
    tratando-se de um sistema Web, é a sanitização de entradas de usuário,
    incluindo via URL (não permitir, por exemplo, um campo ``idAluno'' na URL
    não ser validado com o usuário \textit{logado} e dar acesso a dados de
    outros alunos).

    Para evitar ataques como \textit{buffer-overflow}, é importante manter as
    bibliotecas atualizadas, já que atualizações de segurança são essenciais e
    nem sempre é possível ficar observando qual atualização de cada biblioteca
    terá alguma alteração importante.

    Da parte de infraestrutura, o sistema do CAGR deve ter \textit{backups},
    para evitar perda de informação quando ocorrer alguma invasão com o
    propósito de destruir informações, e \textit{logs} para tanto indentificar
    possíveis atacantes quanto desfazer operações indesejadas.

    \subsection{Custo/Beneficio}

    Apesar de os custos de se realizar um ataque no CAGR não sejam altos, os
    benefícios dificilmente compensariam. Boa parte deles são muito
    direcionados a pessoas específicas em situações específicas. Nas melhores
    hipóteses (do ponto de vista do atacante), se consegue a senha de algum
    usuário que possa utilizar a mesma senha em outros lugares.

    Mesmo um \textit{Denial of Service} não teria tantos benefícios. Inclusive,
    este possivelmente teria um dos custos mais altos, afinal o CAGR é feito
    para suportar acesso de todos os alunos da UFSC --- mais de 46 mil ---
    durante todo o semestre, a ganhos não muito significativos: atrasaria a
    publicação de médias finais, ou iria atrapalhar o processo de matrícula,
    porém nada de concreto se conseguiria.
\end{document}
