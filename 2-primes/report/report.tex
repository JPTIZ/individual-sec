%-------------------------------------------------------------------------------
\documentclass{article}

%-------------------------------------------------------------------------------
% Packages

% Translation
\usepackage[brazilian]{babel}

% General
\usepackage{amsmath}
\usepackage{booktabs}
\usepackage{environ}
\usepackage[margin=1in]{geometry}
\usepackage{minted}
\usepackage{xcolor}

% Other
\usepackage{hyperref}
\usepackage[brazilian]{cleveref} % Must be after hyperref

%-------------------------------------------------------------------------------
% User-commands
\newcommand{\todo}[1]{{\color{red}{#1}}}

\NewEnviron{superframe}{%
    \begin{center}
        \fbox{\setlength{\fboxsep}{1em}\fbox{\parbox{5.5in}{%
            \BODY{}
        }}}
    \end{center}
}

\NewEnviron{answer}{%
    \begin{samepage}
        \begin{solution}
            \BODY{}
        \end{solution}
    \end{samepage}
}

%-------------------------------------------------------------------------------
% Project configs
\title{%
    Segurança em Computação \\
    Trabalho Individual II
}
\author{João Paulo Taylor Ienczak Zanette}
\date{\today}

%-------------------------------------------------------------------------------
\begin{document}
    \maketitle{}

    \begin{superframe}
        \textbf{Quanto ao entregável}

        Você deve entregar no Moodle um único arquivo PDF com os seguintes requisitos:

        \begin{itemize}
            \item Uma seção para \textbf{Números Aleatórios};
            \item Uma seção para \textbf{Números Primos};
            \item Os códigos devem estar no PDF e devidamente documentados;
            \item Referências a cada um dos algoritmos.
        \end{itemize}

        OBS\@: Incluir tabelas, saídas, etc.
    \end{superframe}

    \section{Números Aleatórios}

    Algoritmo escolhido: Xorshift com $(a, b, c) = (13, 7, 17)$.

    \begin{table}[ht]
        \centering
        \begin{tabular}{c l}
            \toprule
            Nº de bits & Tempo (ms) \\
            \midrule
            32 & 0.0045 \\
            64 & 0.0050 \\
            128 & 0.0056 \\
            256 & 0.0053 \\
            512 & 0.0064 \\
            1024 & 0.0054 \\
            2048 & 0.0050 \\
            4096 & 0.0052 \\
            \bottomrule
        \end{tabular}
        \caption{%
            Temporização para o algoritmo de Xorshift. O tempo foi calculado
            utilizando uma média aritmética de 1000 execuções.
        }
    \end{table}

    \section{Números Primos}

    \section{Código}

    \inputminted[lastline=91]{python}{../rng.py}

    \bibliography{}
\end{document}
