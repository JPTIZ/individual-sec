%-------------------------------------------------------------------------------
\documentclass{article}

%-------------------------------------------------------------------------------
% Packages

% Translation
\usepackage[brazilian]{babel}

% General
\usepackage{amsmath}
\usepackage{environ}
\usepackage[margin=1in]{geometry}
\usepackage{xcolor}

% Other
\usepackage{hyperref}
\usepackage[brazilian]{cleveref} % Must be after hyperref

%-------------------------------------------------------------------------------
% User-commands
\newcommand{\todo}[1]{{\color{red}{#1}}}

\NewEnviron{superframe}{%
    \begin{center}
        \fbox{\setlength{\fboxsep}{1em}\fbox{\parbox{5.5in}{%
            \BODY{}
        }}}
    \end{center}
}

\NewEnviron{answer}{%
    \begin{samepage}
        \begin{solution}
            \BODY{}
        \end{solution}
    \end{samepage}
}

%-------------------------------------------------------------------------------
% Project configs
\title{%
    Segurança em Computação \\
    Trabalho Individual II
}
\author{João Paulo Taylor Ienczak Zanette}
\date{\today}

%-------------------------------------------------------------------------------
\begin{document}
    \maketitle{}

    \begin{superframe}
        \textbf{Quanto ao entregável}

        Você deve entregar no Moodle um único arquivo PDF com os seguintes requisitos:

        \begin{itemize}
            \item Uma seção para \textbf{Números Aleatórios};
            \item Uma seção para \textbf{Números Primos};
            \item Os códigos devem estar no PDF e devidamente documentados;
            \item Referências a cada um dos algoritmos.
        \end{itemize}

        OBS\@: Incluir tabelas, saídas, etc.
    \end{superframe}

    \section{Números Aleatórios}

    \section{Números Primos}

    \section{Código}

    \bibliography{}
\end{document}
