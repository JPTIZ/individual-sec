%-------------------------------------------------------------------------------
\documentclass{article}

%-------------------------------------------------------------------------------
% Packages

% Translation
\usepackage[brazilian]{babel}

% General
\usepackage{amsmath}
\usepackage{environ}
\usepackage{graphicx}
\usepackage[margin=1in]{geometry}
\usepackage{minted}
\usepackage{xcolor}

% To-do List
\usepackage{amssymb}
\usepackage{pifont}
\usepackage{enumitem}

% Other
\usepackage{hyperref}
\usepackage[brazilian]{cleveref} % Must be after hyperref


% Custom lists
\newlist{todolist}{itemize}{2}
\setlist[todolist]{label=$\square$}
\newcommand{\done}{%
    \rlap{$\square$}{%
        \raisebox{2pt}{\large\hspace{1pt}\ding{51}
    }}\hspace{-2.5pt}
}

%-------------------------------------------------------------------------------
% User-commands
\newcommand{\todo}[1]{{\color{red}{#1}}}

\NewEnviron{superframe}{%
    \begin{center}
        \fbox{\setlength{\fboxsep}{1em}\fbox{\parbox{5.5in}{%
            \BODY{}
        }}}
    \end{center}
}

\NewEnviron{answer}{%
    \begin{samepage}
        \begin{solution}
            \BODY{}
        \end{solution}
    \end{samepage}
}

\renewcommand{\thesection}{Parte \arabic{section}.}

%-------------------------------------------------------------------------------
% Project configs
\title{%
    Segurança em Computação \\
    Trabalho Individual IV
}
\author{João Paulo Taylor Ienczak Zanette}
\date{\today}

%------------------------------------------------------------------------------
\begin{document}
    \maketitle{}

    %--------------------------------------------------------------------------
    \section{Usando o NMAP}

    \begin{superframe}
        Copie e cole screenshots de telas obtidas na execução dos comandos.
        Explique brevemente a saída obtida em cada um dos comandos das questões
        1, 2, 3 e 4.

        \begin{enumerate}
            \item \texttt{nmap -sV -O 10.1.2.6};
            \item \texttt{nmap -v -A 10.1.2.6};
            \item \texttt{nmap -sS -v --top-ports 10 --reason -oA saidanmap www.ufsc.br};
            \item Crie um comando com opções diferentes das usadas nas questões
                anteriores e explique a saída obtida pelo seu comando.
            \item Responda:
                \begin{enumerate}
                    \item Qual a diferença entre um scan de conexão TCP e um
                        SYN scan?
                    \item Qual questão anterior usa scan de conexão TCP e qual
                        questão usa SYN scan?
                    \item Comente pelo menos uma vulnerabilidade da máquina
                        Owasp Broken, listando a identificação CVE
                        (\url{cve.mitre.org}) da vulnerabilidade.
                \end{enumerate}
        \end{enumerate}
    \end{superframe}

    %--------------------------------------------------------------------------
    \section{Nikto}

    \begin{superframe}
        \begin{enumerate}
            \item Execute o comando \texttt{nikto -host http://10.1.2.6/Wacko -o nikto.html -format htm}.
                \begin{enumerate}
                    \item Copie e cole screenshots de telas obtidas na execução do
                        comando;
                    \item Explique o que mais chamou sua atenção na saída obtida.
                        Explique também alguma vulnerabilidade encontrada nessa
                        aplicação (WackoPicko) que consta no relatório do arquivo
                        muti.html.
                \end{enumerate}
        \end{enumerate}
    \end{superframe}

    %--------------------------------------------------------------------------
    \section{OWASP}

    \begin{superframe}
        \begin{enumerate}
            \item Explique as vulnerabilidades A1, A2, A3 e A7 do documento TOP TEN
                2017:
                \url{https://www.owasp.org/images/7/72/OWASP_Top_10-2017_$28en$29.pdf.pdf};
            \item Faça:
                \begin{enumerate}
                    \item Acesse a aplicação Mutillidae: abra o browser da sua
                        máquina real ou na Kali Linux no site \url{http://<IP da
                        Kali>/mutillidae} e clique em Login. No campo Username,
                        digite \texttt{' or 1=1 -- }. O campo password pode ficar
                        em branco. Copie e cole a tela do seu experimento;
                    \item Explique o resultado obtido e a vulnerabilidade explorada
                        no experimento (pesquise no documento do TOP 10 da OWASP);
                    \item O que pode ser feito para impedir a exploração dessa
                        vulnerabilidade?
                    \item Clique em Logout.
                \end{enumerate}
            \item Repita a inserção da mesma string anterior no seguinte link:
                \url{http://<IP da Kali>/mutillidae/index.php?page=user-info.php};
                \begin{enumerate}
                    \item Explique a vulnerabilidade explorada no experimento
                        (pesquise no documento do TOP 10 da OWASP);
                    \item Copie e cole um screenshot da execução de um experimento;
                    \item O que pode ser feito para impedir a exploração dessa
                        vulnerabilidade?
                \end{enumerate}
            \item Você deve utilizar a ferramenta OWASP ZAP (Zed Attack Proxy) da
                Kali Linux. As ferramentas de scan de web são encontradas no menu
                Kali-Linux -> O3 --- Web Application Analysis -> owasp-zap. Faã um
                scan das vulnerabilidades da aplicação WackoPicko da máquina OWASP
                Broken usando a ferramenta. Faça:
                \begin{enumerate}
                    \item Coloque a URL da aplicação --- \url{http://<IP da
                        OWASP>/WackoPicko} --- e clique em ``Attack''. A análise
                        básica é iniciada. Demora um pouco (de 8 a 10 minutos) e
                        você deve salvar o relatório geral do processo (opção
                        Report -> Generate HTML Report). Os alertas (aba Alerts)
                        vão listando as vulnerabilidades encontradas. Na aba Active
                        Scan é possível ver os requests sendo enviados.
                    \item Comente o experimento e os resultados alcançados.
                \end{enumerate}
            \item Observe a lista de vulnerabilidades da aplicação Mutillidae
                disponível em \url{http://<IP da
                Kali>/mutillidae/index.php?page=./documentation/vulnerabilities.php}.
                Agora você deve escolher duas vulnerabilidades do TOP 10 2017 da
                lista da OWASP e criar uma forma de ataque para cada uma das
                vulnerabilidades escolhidas. Assim, você deve criar dois ataques
                (devem ser diferentes dos ataques das questões 8 e 9). Documente os
                experimentos e mostre funcionando na apresentação. Na apresentação
                você também deve explicar as vulnerabilidades.
        \end{enumerate}
    \end{superframe}
\end{document}
